% Этот шаблон документа разработан в 2014 году
% Данилом Фёдоровых (danil@fedorovykh.ru) 
% для использования в курсе 
% <<Документы и презентации в \LaTeX>>, записанном НИУ ВШЭ
% для Coursera.org: http://coursera.org/course/latex .
% Исходная версия шаблона --- 
% https://www.writelatex.com/coursera/latex/1.1


\documentclass[a4paper,12pt]{article}

\usepackage{cmap}					% поиск в PDF
\usepackage[T2A]{fontenc}			% кодировка
\usepackage[utf8]{inputenc}			% кодировка исходного текста
\usepackage[english,russian]{babel}	% локализация и переносы

%%% Дополнительная работа с математикой
\usepackage{amsmath,amsfonts,amssymb,amsthm,mathtools} % AMS



%\author{Чащин Константин}
\title{Изучаем ТеорВер Stepic}
%\date{\today}


\begin{document} % Конец преамбулы, начало текста.

\maketitle

\section{Элементарная теория вероятностей: случайные события}

\subsection{Обзор}

\subsection{Вероятностная модель эксперимента}
\subsubsection*{Задача}
Из какого количества элементарных событий состоит пространство элементарных событий для следующих испытаний:
\begin{itemize}
	\item 
	    производится выстрел по мишени, представляющей собой 10 концентрических кругов, занумерованных числами от 1 до 10: 11
	\item 
	    три раза подбрасывается игральная кость: 6*6*6=216
	\item 
	    наудачу извлекается одна кость из полной игры домино: 28
\end{itemize}

\subsubsection*{Задача}
Из какого количества элементарных событий состоит пространство элементарных событий для следующего испытания: производится выстрел по мишени, представляющей собой 10 концентрических кругов, занумерованных числами от 1 до 10, а затем столько раз кидается игральная кость, сколько очков выбито на мишени.

Возможные исходы:
\begin{itemize}
	\item выстрел в "молоко"\ - кубик не подбрасывается - количество исходов \(1\)
	\item выстрел в "1"\ - кубик подбрасывается 1 раз - количество исходов \(6\)
	\item выстрел в "2"\ - кубик подбрасывается 2 раза - количество исходов \(6^2\)
	\item выстрел в "3"\ - кубик подбрасывается 3 раза - количество исходов \(6^3\)
	\item выстрел в "4"\ - кубик подбрасывается 4 раза - количество исходов \(6^4\)
	\item выстрел в "5"\ - кубик подбрасывается 5 раз - количество исходов \(6^5\)
	\item выстрел в "6"\ - кубик подбрасывается 6 раз - количество исходов \(6^6\)
	\item выстрел в "7"\ - кубик подбрасывается 7 раз - количество исходов \(6^7\)
	\item выстрел в "8"\ - кубик подбрасывается 8 раз - количество исходов \(6^8\)
	\item выстрел в "9"\ - кубик подбрасывается 9 раз - количество исходов \(6^9\)
	\item выстрел в "10"\ - кубик подбрасывается 10 раз - количество исходов \(6^{10}\)
\end{itemize}
Тогда всего элементарных событий \(1+6^1+6^2+6^3+6^4+6^5+6^6+6^7+6^8+6^9+6^{10} = \sum_{i=0}^{10}6^i = \dfrac{6^{11}-1}{6-1} = 72559411\)

\subsubsection*{Задача}
Из какого количества элементарных событий состоит каждое из следующих случайных событий:
\begin{itemize}
	\item 
сумма двух наудачу выбранных однозначных чисел равна пятнадцати: (элементарное событие — появление пары однозначных чисел (m,n)) Ответ: 4, события (6,9),(7,8),(8,7),(9,6)
    \item 
наудачу выбранная кость из полной игры домино оказалась дублем (элементарное событие — появление кости m:n) Ответ: 7
	\item 
наудачу вырванный листок из нового календаря соответствует тридцатому числу (элементарное событие — появление одного из листков календаря) Ответ: 11
\end{itemize}


\subsection{Вероятностное пространство}
\subsubsection*{Задача}
Студент, изучающий теорию вероятностей, раздобыл отрывной календарь за 2018 год и вырвал в нем наугад одну страницу. Найдите вероятность того, что число на вырванном листке \begin{itemize}
	\item кратно шести: \(\dfrac{59}{365}\)
	\item равно 30: \(\dfrac{11}{365}\)
\end{itemize}

\subsubsection*{Задача}
У кривого игрального кубика грани помечены числами от 1 до 6, а вероятность выпадения грани пропорциональна написанному на ней числу. Событие A означает, что выпало число, меньшее пяти; событие B означает, что выпало нечетное число. Найдите вероятности следующих событий:
\begin{itemize}
	\item \(A \cap B\): \(\dfrac{4}{21}\)
	\item \(A \cup B\): \(\dfrac{15}{21}\)
	\item \(A \setminus B\): \(\dfrac{6}{21}\) 
\end{itemize}
Обозначим \(x\) - вероятность выпадения "1". Тогда вероятность выпадения "2"\ - \(2x\), 
"3"\ - \(3x\), "4"\ - \(4x\), "5"\ - \(5x\), "6"\ - \(6x\). Сумма вероятностей всех возможных исходов равна 1, следовательно \(x + 2x + 3x+4x+5x+6x=21x=1\) и \(x=\dfrac{1}{21}\). 

\subsubsection*{Задача}
Пусть события A и B имеют вероятности 0,5 и 0,7 соответственно. Найдите
\begin{itemize}
	\item Наибольшую вероятность, которую может иметь событие \(A \cup B\): 1.0
	\item Наименьшую вероятность, которую может иметь событие \(A \cup B\): 0.7
	\item Наибольшую вероятность, которую может иметь событие \(A \cap B\): 0.5
	\item Наименьшую вероятность, которую может иметь событие \(A \cap B\): 0.2
\end{itemize}
\subsubsection*{Задача}
Отметьте верные утверждения. Во всех утверждениях \(A\) и \(B\) означают случайные события.
\begin{itemize}
	\item если \(A \subset B\), то \(P(A) \leqslant P(B)\): верно
	\item \(P(A \cap B)\) всегда меньше, чем \(P(A)\): не верно
	\item \(P(A \cap B)\) может быть больше, чем \(P(A)\): не верно
	\item \(P(A \cup B)=P(A)+P(B)-P(A\cap B)\): верно
	\item \(P(A \cap B) \geqslant P(A)+P(B)-1\): верно
	\item \(P(A \cap \overline{B}) = P(A \cup B)\ - P(B)\): верно
	\item событие \(P(A \cup B)\) означает, что произошли оба события \(A\) и \(B\):  верно
	\item если \(P(\overline{A} \cup \overline{B})=1\), то \(P(A \cup B) = P(A)+P(B)\):  верно
\end{itemize}


\subsection{Немного комбинаторики}
\subsubsection*{Задача}
В урне лежат 9 белых шаров и 5 черных. Из урны одновременно извлекаются два шара. Найдите вероятности следующих событий
\begin{itemize}
	\item Извлеченные шары одного цвета: \(\dfrac{46}{91}\)
	\item Извлеченные шары разных цветов: \(\dfrac{45}{91}\)
\end{itemize}

\subsubsection*{Задача}
На карточках написаны все трехзначные числа, каждое по одному разу. Сколькими способами можно выбрать три карточки с четной суммой.

Сумма будет чётной, если все три карточки чётные или одна чётная и две нечётные. Всего карточек 900 (100...999). Чётных и нечётных ровно половина. Тогда три чётных карточек можно выбрать \(\binom{450}{3}=\dfrac{450*449*448}{3*2} = 15086400\) способами. Одну чётную и две нечётных  карточек можно выбрать \(\binom{450}{1}\binom{450}{2}=45461250\) способами. Всего \(15086400+45461250=60547650\) способов.



\subsubsection*{Задача}
В чемпионате России по футболу участвует 16 команд. Назовем итоги двух первенств похожими, если в них совпадают обладатели золотых, серебряных и бронзовых медалей; команды занявшие четвертые места (они получают право играть в европейских кубках), команды занявшие 13-е места, команды занявшие 14-е места (эти команды играют стыковые матчи); а также команды напрямую покидающие премьер-лигу (т.е. команды, занявшие последнее и предпоследнее места). Сколько существует попарно непохожих итогов чемпионата?

На первое место можно поставить команду шестнадцатью способами, на второе 15-ю, на третье 14-ю, на четвёртое 13-ю, на тринадцатое 12-ю, на четырнадцатое 11-ю. По условию задачи для команд занявших предпоследнее и последнее места порядок не важен, поэтому остаётся ещё выбрать 2 команды из оставшихся 10 - это \(\binom{10}{2}\) способов. Всего способов  \(\binom{10}{2}*16*15*14*13*12*11=259459200\)

\subsubsection*{Задача}
В программе к экзамену по теории вероятностей 75 вопросов. Студент знает 50 из них. В билете 3 вопроса. Найдите вероятность того, что студент знает хотя бы два вопроса из вытянутого им билета. В ответе приведите обыкновенную дробь.

Всего вариантов составить билеты по три вопроса из 75 вопросов \(\binom{75}{3}=67525\).

Вариантов что в билете будут два вопроса из 50 (знает ответ) и один из 25 (не знает ответа) \(\binom{50}{2}\binom{25}{1}=30625\).

Вариантов что в билете будут три вопроса из 50 (знает ответ)  \(\binom{50}{3}=19600\).

Подходят оба варианта, тогда искомая вероятность \(\dfrac{19600+30625}{67525}=\dfrac{2009}{2701}\).


\subsubsection*{Задача}
Пусть \(n \geqslant 3\). Шарики занумерованы числами от 1 до n. Найдите количество способов эти n шариков разместить в n разных ящиков так, чтобы ровно один ящик оказался пустым.

Выбираем пустой ящик - n спосбов.

Выбираем ящик для двух шариков - (n-1) способов.

Выбираем два шарика - \(\binom{n}{2}\) способов.

Остальные (n-2) шарика переставляем по (n-2) ящикам - \((n-2)!\).

Ответ получается перемножением \(n(n-1)\binom{n}{2}(n-2)!=\dfrac{n(n-1)}{2}n!\)

\subsubsection*{Задача}
Из колоды в 52 карты наугад взяли 6 карт. Найдите вероятности событий
\begin{itemize}
	\item среди выбранных карт по три карты двух разных мастей: \(\dfrac{490776}{20358520}\)
	\item среди выбранных карт не более двух бубновых карт: \(\dfrac{17163042}{20358520}\)
\end{itemize}

Всего вариантов \(\binom{52}{6}=20358520\)

Первый вопрос: выбираем две масти \(\binom{4}{2}\). Для каждой масти выбираем три карты \(\binom{13}{3}\). Тогда число удовлетворительных результатов \(\binom{4}{2}\binom{13}{3}^2=490776\).

Второй вопрос: не выбрать бубновую масть \(\binom{39}{6}=3262623\) способов. Выбрать только одну карту бубновой масти \(\binom{13}{1}\binom{39}{5}=7484841\) способов. Выбрать две карты бубновой масти \(\binom{13}{2}\binom{39}{4}=6415578\) способов. Всего \(3262623+7484841+6415578=17163042\).


\subsection{Условная вероятность}
\subsubsection*{Задача}
В урне 11 красных, 10 синих и 9 зеленых шаров. Из нее последовательно вынимают три шара. Найдите вероятность того, что первый шар окажется красным, второй — синим, а третий — зеленым. В ответе приведите обыкновенную дробь.

\paragraph{Первый способ}
Элементарные события - упорядоченные тройки шаров. Всего таких троек \(A_{30}^3=30*29*28=24360\). Количество способов выбрать на первое место красный шар \(\binom{11}{1}=11\), на второе место синий шар \(\binom{10}{1}=10\), на третье место зелёный шар \(\binom{9}{1}=9\). Тогда вероятность равна \(\dfrac{11*10*9}{30*29*28}=\dfrac{33}{812}\).

\paragraph{Второй способ}
Обозначим события: A - первый выташенный шар красный, B - второй вытащенный шар синий, C - третий вытащенный шар зелёный. Надо найти вероятность того, что одновременно произошли все три события A, B и C.
\[P(A\cap B \cap C)=P(C|(A \cap B))P(A \cap B)\]
\[P(A \cap B)=P(B|A)P(A)\]

Событие A - все тройки с первым красным шаром. Таких троек \(11*29*28\). Тогда \[P(A)=\dfrac{11*29*28}{A_{20}^3}=\dfrac{11*29*28}{30*29*28}=\dfrac{11}{30}\]

Событие D при условии A, это все тройки, где первый шар уже выбран и он красный, а второй синий. Таких троек \(11*10*28\). Тогда
\[P(B|A)=\dfrac{11*10*28}{11*29*28}=\dfrac{10}{29}\]
и
\[P(A \cap B)=P(B|A)P(A)=\dfrac{11}{30}\dfrac{10}{29}=\dfrac{11*10}{30*29}\]

Событие C при условии A и B, это все тройки, где первый шар уже выбран и он красный, второй шар уже выбран и он синий, а третий шар зелёный синий. Таких троек \(11*10*9\). Тогда
\[P(C|(A \cap B))=\dfrac{11*10*9}{11*10*28}=\dfrac{9}{28}\]

Окончательно получаем 
\[P(A\cap B \cap C)=P(C|(A \cap B))P(A \cap B)=\dfrac{11*10*9}{30*29*28}=\dfrac{33}{812}\]

\subsubsection*{Задача}
Четыре человека A, B, C и D становятся в очередь в случайном порядке. Найдите

\subsection{Теорема Баейса}
\subsection{Независимые события}
\subsection{Схема Бернулли}
\subsection{Краткие сведения из математического анализа}


\section{Элементарная теория вероятностей: случайные величины}




\end{document} % Конец текста.

