% Этот шаблон документа разработан в 2014 году
% Данилом Фёдоровых (danil@fedorovykh.ru) 
% для использования в курсе 
% <<Документы и презентации в \LaTeX>>, записанном НИУ ВШЭ
% для Coursera.org: http://coursera.org/course/latex .
% Исходная версия шаблона --- 
% https://www.writelatex.com/coursera/latex/1.1


\documentclass[a4paper,12pt]{article}

\usepackage{cmap}					% поиск в PDF
\usepackage[T2A]{fontenc}			% кодировка
\usepackage[utf8]{inputenc}			% кодировка исходного текста
\usepackage[english,russian]{babel}	% локализация и переносы

%%% Дополнительная работа с математикой
\usepackage{amsmath,amsfonts,amssymb,amsthm,mathtools} % AMS



%\author{Чащин Константин}
\title{Изучаем ТеорВер Stepic}
%\date{\today}


\begin{document} % Конец преамбулы, начало текста.

\maketitle

\section{Элементарная теория вероятностей: случайные события}

\subsection{Обзор}

\subsection{Вероятностная модель эксперимента}
\subsubsection*{Задача}
Из какого количества элементарных событий состоит пространство элементарных событий для следующих испытаний:
\begin{itemize}
	\item 
	    производится выстрел по мишени, представляющей собой 10 концентрических кругов, занумерованных числами от 1 до 10: 11
	\item 
	    три раза подбрасывается игральная кость: 6*6*6=216
	\item 
	    наудачу извлекается одна кость из полной игры домино: 28
\end{itemize}

\subsubsection*{Задача}
Из какого количества элементарных событий состоит пространство элементарных событий для следующего испытания: производится выстрел по мишени, представляющей собой 10 концентрических кругов, занумерованных числами от 1 до 10, а затем столько раз кидается игральная кость, сколько очков выбито на мишени.

Возможные исходы:
\begin{itemize}
	\item выстрел в "молоко"\ - кубик не подбрасывается - количество исходов \(1\)
	\item выстрел в "1"\ - кубик подбрасывается 1 раз - количество исходов \(6\)
	\item выстрел в "2"\ - кубик подбрасывается 2 раза - количество исходов \(6^2\)
	\item выстрел в "3"\ - кубик подбрасывается 3 раза - количество исходов \(6^3\)
	\item выстрел в "4"\ - кубик подбрасывается 4 раза - количество исходов \(6^4\)
	\item выстрел в "5"\ - кубик подбрасывается 5 раз - количество исходов \(6^5\)
	\item выстрел в "6"\ - кубик подбрасывается 6 раз - количество исходов \(6^6\)
	\item выстрел в "7"\ - кубик подбрасывается 7 раз - количество исходов \(6^7\)
	\item выстрел в "8"\ - кубик подбрасывается 8 раз - количество исходов \(6^8\)
	\item выстрел в "9"\ - кубик подбрасывается 9 раз - количество исходов \(6^9\)
	\item выстрел в "10"\ - кубик подбрасывается 10 раз - количество исходов \(6^{10}\)
\end{itemize}
Тогда всего элементарных событий \(1+6^1+6^2+6^3+6^4+6^5+6^6+6^7+6^8+6^9+6^{10} = \sum_{i=0}^{10}6^i = \dfrac{6^{11}-1}{6-1} = 72559411\)

\subsubsection*{Задача}
Из какого количества элементарных событий состоит каждое из следующих случайных событий:
\begin{itemize}
	\item 
сумма двух наудачу выбранных однозначных чисел равна пятнадцати: (элементарное событие — появление пары однозначных чисел (m,n)) Ответ: 4, события (6,9),(7,8),(8,7),(9,6)
    \item 
наудачу выбранная кость из полной игры домино оказалась дублем (элементарное событие — появление кости m:n) Ответ: 7
	\item 
наудачу вырванный листок из нового календаря соответствует тридцатому числу (элементарное событие — появление одного из листков календаря) Ответ: 11
\end{itemize}


\subsection{Вероятностное пространство}
\subsubsection*{Задача}
Студент, изучающий теорию вероятностей, раздобыл отрывной календарь за 2018 год и вырвал в нем наугад одну страницу. Найдите вероятность того, что число на вырванном листке \begin{itemize}
	\item кратно шести: \(\dfrac{59}{365}\)
	\item равно 30: \(\dfrac{11}{365}\)
\end{itemize}

\subsubsection*{Задача}
У кривого игрального кубика грани помечены числами от 1 до 6, а вероятность выпадения грани пропорциональна написанному на ней числу. Событие A означает, что выпало число, меньшее пяти; событие B означает, что выпало нечетное число. Найдите вероятности следующих событий:
\begin{itemize}
	\item \(A \cap B\): \(\dfrac{4}{21}\)
	\item \(A \cup B\): \(\dfrac{15}{21}\)
	\item \(A \setminus B\): \(\dfrac{6}{21}\) 
\end{itemize}
Обозначим \(x\) - вероятность выпадения "1". Тогда вероятность выпадения "2"\ - \(2x\), 
"3"\ - \(3x\), "4"\ - \(4x\), "5"\ - \(5x\), "6"\ - \(6x\). Сумма вероятностей всех возможных исходов равна 1, следовательно \(x + 2x + 3x+4x+5x+6x=21x=1\) и \(x=\dfrac{1}{21}\). 

\subsubsection*{Задача}
Пусть события A и B имеют вероятности 0,5 и 0,7 соответственно. Найдите
\begin{itemize}
	\item Наибольшую вероятность, которую может иметь событие \(A \cup B\): 1.0
	\item Наименьшую вероятность, которую может иметь событие \(A \cup B\): 0.7
	\item Наибольшую вероятность, которую может иметь событие \(A \cap B\): 0.5
	\item Наименьшую вероятность, которую может иметь событие \(A \cap B\): 0.2
\end{itemize}
\subsubsection*{Задача}
Отметьте верные утверждения. Во всех утверждениях \(A\) и \(B\) означают случайные события.
\begin{itemize}
	\item если \(A \subset B\), то \(P(A) \leqslant P(B)\): верно
	\item \(P(A \cap B)\) всегда меньше, чем \(P(A)\): не верно
	\item \(P(A \cap B)\) может быть больше, чем \(P(A)\): не верно
	\item \(P(A \cup B)=P(A)+P(B)-P(A\cap B)\): верно
	\item \(P(A \cap B) \geqslant P(A)+P(B)-1\): верно
	\item \(P(A \cap \overline{B}) = P(A \cup B)\ - P(B)\): верно
	\item событие \(P(A \cup B)\) означает, что произошли оба события \(A\) и \(B\):  верно
	\item если \(P(\overline{A} \cup \overline{B})=1\), то \(P(A \cup B) = P(A)+P(B)\):  верно
\end{itemize}


\subsection{Немного комбинаторики}
\subsubsection*{Задача}
В урне лежат 9 белых шаров и 5 черных. Из урны одновременно извлекаются два шара. Найдите вероятности следующих событий
\begin{itemize}
	\item Извлеченные шары одного цвета: \(\dfrac{46}{91}\)
	\item Извлеченные шары разных цветов: \(\dfrac{45}{91}\)
\end{itemize}

\subsubsection*{Задача}
На карточках написаны все трехзначные числа, каждое по одному разу. Сколькими способами можно выбрать три карточки с четной суммой.

Сумма будет чётной, если все три карточки чётные или одна чётная и две нечётные. Всего карточек 900 (100...999). Чётных и нечётных ровно половина. Тогда три чётных карточек можно выбрать \(\binom{450}{3}=\dfrac{450*449*448}{3*2} = 15086400\) способами. Одну чётную и две нечётных  карточек можно выбрать \(\binom{450}{1}\binom{450}{2}=45461250\) способами. Всего \(15086400+45461250=60547650\) способов.



\subsubsection*{Задача}
В чемпионате России по футболу участвует 16 команд. Назовем итоги двух первенств похожими, если в них совпадают обладатели золотых, серебряных и бронзовых медалей; команды занявшие четвертые места (они получают право играть в европейских кубках), команды занявшие 13-е места, команды занявшие 14-е места (эти команды играют стыковые матчи); а также команды напрямую покидающие премьер-лигу (т.е. команды, занявшие последнее и предпоследнее места). Сколько существует попарно непохожих итогов чемпионата?

На первое место можно поставить команду шестнадцатью способами, на второе 15-ю, на третье 14-ю, на четвёртое 13-ю, на тринадцатое 12-ю, на четырнадцатое 11-ю. По условию задачи для команд занявших предпоследнее и последнее места порядок не важен, поэтому остаётся ещё выбрать 2 команды из оставшихся 10 - это \(\binom{10}{2}\) способов. Всего способов  \(\binom{10}{2}*16*15*14*13*12*11=259459200\)

\subsubsection*{Задача}
В программе к экзамену по теории вероятностей 75 вопросов. Студент знает 50 из них. В билете 3 вопроса. Найдите вероятность того, что студент знает хотя бы два вопроса из вытянутого им билета. В ответе приведите обыкновенную дробь.

Всего вариантов составить билеты по три вопроса из 75 вопросов \(\binom{75}{3}=67525\).

Вариантов что в билете будут два вопроса из 50 (знает ответ) и один из 25 (не знает ответа) \(\binom{50}{2}\binom{25}{1}=30625\).

Вариантов что в билете будут три вопроса из 50 (знает ответ)  \(\binom{50}{3}=19600\).

Подходят оба варианта, тогда искомая вероятность \(\dfrac{19600+30625}{67525}=\dfrac{2009}{2701}\).


\subsubsection*{Задача}
Пусть \(n \geqslant 3\). Шарики занумерованы числами от 1 до n. Найдите количество способов эти n шариков разместить в n разных ящиков так, чтобы ровно один ящик оказался пустым.

Выбираем пустой ящик - n спосбов.

Выбираем ящик для двух шариков - (n-1) способов.

Выбираем два шарика - \(\binom{n}{2}\) способов.

Остальные (n-2) шарика переставляем по (n-2) ящикам - \((n-2)!\).

Ответ получается перемножением \(n(n-1)\binom{n}{2}(n-2)!=\dfrac{n(n-1)}{2}n!\)

\subsubsection*{Задача}
Из колоды в 52 карты наугад взяли 6 карт. Найдите вероятности событий
\begin{itemize}
	\item среди выбранных карт по три карты двух разных мастей: \(\dfrac{490776}{20358520}\)
	\item среди выбранных карт не более двух бубновых карт: \(\dfrac{17163042}{20358520}\)
\end{itemize}

Всего вариантов \(\binom{52}{6}=20358520\)

Первый вопрос: выбираем две масти \(\binom{4}{2}\). Для каждой масти выбираем три карты \(\binom{13}{3}\). Тогда число удовлетворительных результатов \(\binom{4}{2}\binom{13}{3}^2=490776\).

Второй вопрос: не выбрать бубновую масть \(\binom{39}{6}=3262623\) способов. Выбрать только одну карту бубновой масти \(\binom{13}{1}\binom{39}{5}=7484841\) способов. Выбрать две карты бубновой масти \(\binom{13}{2}\binom{39}{4}=6415578\) способов. Всего \(3262623+7484841+6415578=17163042\).


\subsection{Условная вероятность}
\subsubsection*{Задача}
В урне 11 красных, 10 синих и 9 зеленых шаров. Из нее последовательно вынимают три шара. Найдите вероятность того, что первый шар окажется красным, второй — синим, а третий — зеленым. В ответе приведите обыкновенную дробь.

\paragraph{Первый способ}
Элементарные события - упорядоченные тройки шаров. Всего таких троек \(A_{30}^3=30*29*28=24360\). Количество способов выбрать на первое место красный шар \(\binom{11}{1}=11\), на второе место синий шар \(\binom{10}{1}=10\), на третье место зелёный шар \(\binom{9}{1}=9\). Тогда вероятность равна \(\dfrac{11*10*9}{30*29*28}=\dfrac{33}{812}\).

\paragraph{Второй способ}
Обозначим события: A - первый выташенный шар красный, B - второй вытащенный шар синий, C - третий вытащенный шар зелёный. Надо найти вероятность того, что одновременно произошли все три события A, B и C.
\[P(A\cap B \cap C)=P(C|(A \cap B))P(A \cap B)\]
\[P(A \cap B)=P(B|A)P(A)\]

Событие A - все тройки с первым красным шаром. Таких троек \(11*29*28\). Тогда \[P(A)=\dfrac{11*29*28}{A_{20}^3}=\dfrac{11*29*28}{30*29*28}=\dfrac{11}{30}\]

Событие D при условии A, это все тройки, где первый шар уже выбран и он красный, а второй синий. Таких троек \(11*10*28\). Тогда
\[P(B|A)=\dfrac{11*10*28}{11*29*28}=\dfrac{10}{29}\]
и
\[P(A \cap B)=P(B|A)P(A)=\dfrac{11}{30}\dfrac{10}{29}=\dfrac{11*10}{30*29}\]

Событие C при условии A и B, это все тройки, где первый шар уже выбран и он красный, второй шар уже выбран и он синий, а третий шар зелёный синий. Таких троек \(11*10*9\). Тогда
\[P(C|(A \cap B))=\dfrac{11*10*9}{11*10*28}=\dfrac{9}{28}\]

Окончательно получаем 
\[P(A\cap B \cap C)=P(C|(A \cap B))P(A \cap B)=\dfrac{11*10*9}{30*29*28}=\dfrac{33}{812}\]

\subsubsection*{Задача}
Четыре человека A, B, C и D становятся в очередь в случайном порядке. Найдите:
\begin{itemize}
	\item условную вероятность того, что A первый, если B последний: \(\frac{1}{3}\)
    \item условную вероятность того, что A первый, если A не последний: \(\frac{1}{3}\)
	\item условную вероятность того, что A первый, если B не последний: \(\frac{2}{9}\)
	\item условную вероятность того, что A первый, если B стоит в очереди позже A:  \(\frac{1}{2}\)
	\item условную вероятность того, что A стоит в очереди раньше B, если известно, что A раньше C:  \(\frac{2}{3}\)
\end{itemize}
\paragraph{1}
Событие X - A первый, Y - B последний. Найти \(P(X|Y)=\dfrac{P(X\cap Y)}{P(Y)}\). Всего четвёрок 4!. Событие Y - все четвёрки, у которых на последнем месте B. Таких четвёрок 3!. Среди четвёрок события Y есть четвёрки события X, это те четвёрки у которых на первом месте A. Таких четвёрок 2! - два места фиксированы а остальные два переставляем произвольно. Тогда искомая вероятность \(P(X|Y)=\dfrac{P(X\cap Y)}{P(Y)}=\dfrac{2!}{3!}=\dfrac{1}{3}\).

\paragraph{2}
Событие X - A первый, Y - A не последний. Найти \(P(X|Y)=\dfrac{P(X\cap Y)}{P(Y)}\).
Событие Y - все четвёрки, у которых A на первом, втором или третьем месте. Таких четвёрок \(3*3!\). Среди этих четвёрок есть те, у которых A не первом месте, таких троек \(3!\). Значит искомая вероятность \(P(X|Y)=\dfrac{1}{3}\).

\paragraph{3}
Событие X - A первый, Y - B не последний. Найти \(P(X|Y)=\dfrac{P(X\cap Y)}{P(Y)}\).  Событие Y - все четвёрки, у которых B на первом, втором или третьем месте. Таких четвёрок \(3*3!\). Среди этих четвёрок нам подходят те, где на первом месте стоит A. Таких четвёрок: на первом месте A, на вторам B - таких \(2!\) или на первом месте A, на третьем B - таких \(2!\), всего \(2*2!\).. Тогда искомая вероятнсть \(P(X|Y)=\dfrac{2*2!}{3*3!}=\dfrac{2}{9}\).

\paragraph{4}
Событие X - A первый, Y - B стоит в очереди позже A. Найти \(P(X|Y)=\dfrac{P(X\cap Y)}{P(Y)}\).  Событие Y - четвёрки у которых А на первом месте (таких 3!), плюс четвёрки у которых А на втором месте а B на третьем или четвёртом месте (таких 2*2!), плюс четвёрки у которых A на третьем месте а B на четвёртом месте (таких 2!). Таких четвёрок \(3*3!\). Среди этих четвёрок нам подходят те, где на первом месте стоит A. Событию X благоприятствуют только четвёрки у которых А на первом месте (таких 3!). Тогда искомая вероятнсть \(P(X|Y)=\dfrac{3!}{3!+2*2!+2!}=\dfrac{3!}{3!+3!}=\dfrac{1}{2}\).

\paragraph{5}
Событие X - A стоит в очереди раньше B, Y - A стоит в очереди раньше C. Найти \(P(X|Y)=\dfrac{P(X\cap Y)}{P(Y)}\).
 Событие Y - четвёрки у которых A на первом месте (таких 3!),
 плюс четвёрки у которых A на втором месте и С на третьем месте и B на четвёртом месте (таких 1),
  плюс четвёрки у которых A на втором месте и C на третьем и B на первом месте (таких 1),
    плюс четвёрки у которых A на втором месте и C на четвёртом месте и B на третьем (таких 1),
    плюс четвёрки у которых A на втором месте и C на четвёртом месте и B на первом (таких 1),
     плюс четвёрки у которых A на третьем месте и C на четвёртом месте  (таких 2!). Событию X благоприятствуют только четвёрки у которых А на первом месте (таких 3! + 1 + 1). Тогда искомая вероятнсть \(P(X|Y)=\dfrac{3! + 1+1}{3!+ 1 + 1+1+1+2!}=\dfrac{8}{12}=\dfrac{2}{3}\).



\subsubsection*{Задача}
Игральную кость бросают до тех пор пока не выпадет единица. Найдите вероятность того, что это случилось на втором бросании, если известно, что для этого потребовалось четное число бросаний.
\paragraph{Решение}
Событие A - на втором бросании выпала 1, событие B - потребовалось чётное число бросаний. \(P(A|B)=\dfrac{P(A\cap B)}{P(B)}\). Событие B - потребовалось 2,4,6... бросаний. Вероятность что потребовалось 2-х бросания \(\dfrac{5}{6}\dfrac{1}{6}\). Вероятность что потребовалось 4-х бросания \(\left(\dfrac{5}{6}\right)^3\dfrac{1}{6}\). Вероятность что потребовалось k бросаний (k - чётное) \(\left(\dfrac{5}{6}\right)^{k-1}\dfrac{1}{6}\). Тогда вероятность чётного числа бросаний \(P(B)=\sum_{k=2,4,6...}^{\infty}\left(\dfrac{5}{6}\right)^{k-1}\dfrac{1}{6}=\dfrac{5}{6}\dfrac{1}{6}+\left(\dfrac{5}{6}\right)^3\dfrac{1}{6}+ \ldots+\left(\dfrac{5}{6}\right)^{k-1}\dfrac{1}{6}+ \ldots=\dfrac{1}{6} \cdot \dfrac{5}{6}\cdot\left(1+\left(\dfrac{5}{6}\right)^2+\left(\dfrac{5}{6}\right)^4+\left(\dfrac{5}{6}\right)^6+\ldots\right)=\dfrac{1}{6}\cdot\dfrac{5}{6}\cdot\dfrac{1}{1-\left(\dfrac{5}{6}\right)^2}=\dfrac{5}{36}\cdot\dfrac{36}{11}=\dfrac{5}{11}\). 

Событие \(A \cap B\)  означает что потребовалось 2 бросания и его вероятность содержится в сумме \( P(A\cap B)=\dfrac{1}{6}\cdot\dfrac{5}{6}=\dfrac{5}{36}\).

Тогда искомая вероятность \(P(A|B)=\dfrac{P(A\cap B)}{P(B)}=\dfrac{\dfrac{5}{36}}{\dfrac{5}{11}}=\dfrac{11}{36}\).




\subsection{Теорема Баейса}
\subsubsection*{Задача}
Из полного набора костей домино взята одна кость. Найдите вероятность того, что наудачу взятую вторую кость можно приставить к первой по правилам домино. В качестве ответа приведите обыкновенную дробь

\paragraph{Решение}
Элементарный исход - упорядоченная пара костей домино при выборе без возвращения. Всего таких упорядоченных пар \(28*27=756\). 

Обозначим A событие, что наудачу взятую вторую кость можно приставить к первой по правилам домино. Разобъём множество элементарных событий B на две непересекающиеся части: \(B_1\) первая кость - дубль, \(B_2\) первая кость не дубль. Тогда по формуле полной вероятности \( P(A) = P(A|B_1) \cdot P(B_1) + P(A|B_2) \cdot P(B_2) \).
Вероятность вытащить первой костью дубль \[ P(B_1) = \dfrac{7}{28}=\dfrac{1}{4} \]
Вероятность вытащить первой костью не дубль \[ P(B_2) = \dfrac{21}{28}=\dfrac{3}{4} \]
Вероятность того что вторую кость можно приставить к первой если первая кость дубль \( P(A|B_1) = \dfrac{6}{27} \), так как первая кость зафиксировала количество очков а в оставшихся 27 костях лишь 6 костей с таким количеством очков. 
Вероятность того что вторую кость можно приставить к первой если первая кость не дубль \( P(A|B_1) = 2\cdot\dfrac{6}{27} \), так как первая кость зафиксировала два количества очков а в оставшихся 27 костях 6 костей с каждым количеством очков.
Тогда \( P(A) = P(A|B_1) \cdot P(B_1) + P(A|B_2) \cdot P(B_2) = \dfrac{6}{27} \cdot \dfrac{1}{4} + \dfrac{12}{27} \cdot \dfrac{3}{4} = \dfrac{7}{18} \)



\subsubsection*{Задача}
В урне находится \(n\) шаров, некоторые из них белые. Событие \(A_k\) при \(k=0,1,…,n\) состоит в том, что в урне ровно \(k\) белых шаров. Предположим, что все эти события равновероятны, т.е. \(P(A_0)=P(A_1)=\ldots=P(A_n)=\dfrac{1}{n+1} \). Пусть B — событие, состоящее в том, что наугад взятый шар из урны — белый. Найдите \( P(A_k|B) \).
\paragraph{Решение}
Элементарный исход - вытащен шар из корзины в которой от 0 до n белых шаров. Тогда всё множество элементарных исходов A можно разбить на непересекающиеся подмножества \(A_k\) и применить теорему Байеса:
\[ P(A_k|B) = \dfrac{P(B|A_k) \cdot P(A_k) }{P(B)} \]
\[ P(B) = \sum_{k=0}^{n} \left(P(B|A_k) \cdot P(A_k)\right) \]


\( P(B) = \sum_{k=0}^{n} \left(P(B|A_k) \cdot P(A_k)\right) = P(B|A_0) \cdot P(A_0) + P(B|A_1) \cdot P(A_1) + \ldots + P(B|A_n) \cdot P(A_n) \). 

\( P(B|A_k) = \dfrac{k}{n} \)

Тогда \( P(B) = \dfrac{0}{n} \cdot \dfrac{1}{n+1} + \dfrac{1}{n} \cdot \dfrac{1}{n+1} + \ldots + \dfrac{n}{n} \cdot \dfrac{1}{n+1} = \dfrac{n}{n} \cdot \dfrac{1}{n+1} \cdot \left( 1+\ldots + n \right) = \dfrac{1}{2} \).

и \(  P(A_k|B) = \dfrac{P(B|A_k) \cdot P(A_k) }{P(B)} = \dfrac{\dfrac{k}{n} \cdot \dfrac{1}{n+1}}{\dfrac{1}{2}} = \dfrac{2k}{n(n+1)}  \)

\subsubsection*{Задача}
В понедельник, после двух выходных, токарь Григорий вытачивает левовинтовые шурупы вместо обычных правовинтовых с вероятностью 0,5. Во вторник этот показатель снижается до 0,2. В остальные дни недели Григорий ударно трудится, и процент брака среди изготавливаемых им шурупов составляет 10\%. При проверке недельной партии шурупов, выточенных Григорием, случайно выбранный шуруп оказался дефектным. Какова вероятность того, что шуруп изготовлен в понедельник, если известно, что в понедельник он вытачивает в два раза меньше шурупов, чем в каждый из остальных рабочих дней? В качестве ответа приведите обыкновенную дробь.
\paragraph{Решение}
Элементарное событие - выбран шуруп из множества всех шурупов. Разобъём множество всех шурупов A на непересекающиеся множества шурупов сделанных в понедельник \(A_1\), во вторник \(A_2\), в среду или четверг или пятницу \(A_3\). Применим теорему Байеса.
\[ P(A_1|B) = \dfrac{P(B|A_1) \cdot P(A_1) }{P(B)} \]
\[ P(B) = \sum_{k=1}^{3} \left(P(B|A_k) \cdot P(A_k)\right) \]
Всего сделано шурупов \( x + 4 \cdot 2 \cdot x  = 9x\). где x - число шурупов сдеданных в понедельник.

Тогда \( P(A_1) = \dfrac{1}{9} \),
\( P(A_2) = \dfrac{2}{9} \),
\( P(A_3) = \dfrac{6}{9} \).

По условию задачи \( P(B|A_1) = \dfrac{1}{2} \), 
\( P(B|A_2) = \dfrac{1}{5} \), \( P(B|A_3) = \dfrac{1}{10} \) 

И получаем \( P(A_1|B) = \dfrac{P(B|A_1) \cdot P(A_1) }{P(B)} = \dfrac{\dfrac{1}{2} \cdot \dfrac{1}{9}}{\dfrac{1}{2} \cdot \dfrac{1}{9} + \dfrac{1}{5} \cdot \dfrac{2}{9} + \dfrac{1}{10} \cdot \dfrac{6}{9}} = \dfrac{1}{3} \)


\subsubsection*{Задача}
Даны натуральные числа \(m\) и \(n\), причем \(m<n\). Из чисел \(1,2,…,n\) последовательно выбирают наугад
два различных числа. Найдите вероятность того, что разность между первым выбранным числом и вторым будет не меньше \(m\).
\paragraph{Решение}
Элементарный исход - упорядоченная пара чисел. Количество способов выбрать первое число \( \dfrac{1}{n} \), количество способов выбрать второе число \( \dfrac{1}{n+1} \). Всего элементарных исходов \( \dfrac{1}{n(n+1)} \).

Обозначим первое выбранное число \(a\), второе число \(b\). Если выбрано число \(a\), то благоприятному исходу удовлетворяет такое число \(b\) что \( (a-b) \geqslant m\) или \( b \leqslant (a - m) \). Вероятность при условии что \(a=i\) выбрать "хорошее" число \(b\) равна 0 при \(i \leqslant m\) и \( P(b_{good}|a=i) = \dfrac{i-m}{n-1} \) при \( i > m \).

Тогда вероятность \( P(b_{good}) = P(b_{good}|a=1) \cdot P(a=1) + P(b_{good}|a=2) \cdot P(a=2) + P(b_{good}|a=3) \cdot P(a=3) + \ldots + P(b_{good}|a=n) \cdot P(a=n) = \sum_{i > m}^{n} \left( P(b_{good}|a=i) \cdot P(a=i) \right) \). Подставляя ранее найденные вероятности выбора первого числа \(a\) равной \(\dfrac{1}{n}\) и \( P(b_{good}|a=i) = \dfrac{i-m}{n-1} \) получим что искомая вероятность выбора "хорошего" числа \(b\) равна 

\( \sum_{i > m}^{n} \left( P(b_{good}|a=i) \cdot P(a=i) \right) =
 \sum_{i > m}^{n} \left( \dfrac{i-m}{n-1} \cdot \dfrac{1}{n} \right) = \) 

\(\dfrac{1}{n} \cdot \dfrac{1}{n-1} \cdot \left( 1+2+\ldots + (n-m) \right) = \dfrac{(n-m)(n-m+1)}{2n(n-1)} \).


\subsubsection*{Задача}
Дано натуральное число \(n<52\). Из тщательно перемешанной колоды в \(52\) карты одновременно были взяты \(n\) карт. На одну из этих n карт посмотрели, она оказалась тузом. После этого она возвращается в набор взятых карт и эти \(n\) карт перемешиваются. После этого из них выбирается одна карта и открывается. Найдите вероятность того, что открытая карта является тузом.
\paragraph{Решение Способ 1}
Обозначим A - повторно из набора \(n\) карт выбирается туз, \(B_1\) повторно из набора \(n\) карт выбирается та-же самая карта, \(B_2\) повторно из набора \(n\) карт выбирается другая карта. Тогда по формуле полной вероятности \[ P(A) = P(A|B_1) \cdot P(B_1) + P(A|B_2) \cdot P(B_2) \].

Вероятность выбрать ту-же карту из набора \(n\) карт \(P(B_1)\) равна \( \dfrac{1}{n} \), а вероятность что эта карта при этом условии окажется тузом \(P(A|B_1)\) равна 1. Поэтому первое слагаемое равно \( \dfrac{1}{n} \).

Вероятность выбрать другую карту из набора \(n\) карт \(P(B_2)\) равна \( \dfrac{n-1}{n} \). Обозначим туз который видели в наборе туз1. Рассмотрим подробнее вероятность вытащить какой-то конкретный туз2 при этом условии \(P(A|B_2)\) (это не туз1!). Эту вероятность можно найти по формуле полной вероятности рассмотрев два случая: туз2 попал в набор и туз2 не попал в набор. Если туз2 не попал в набор, то вероятность вытащить его равна 0. Тогда осталось найти вероятность того что этот туз2 попал в набор и умножить на вероятность вытащить туз2 из набора. Известно, что в наборе из \(n\) карт есть туз1. Таких наборов в колоде из 52 карт существует \( \binom{51}{n-1} \). Наборов в которых есть туз2 существует \( \binom{50}{n-2} \). Тогда вероятность попадания туз2 в набор равна \( \dfrac{\binom{50}{n-2}}{\binom{51}{n-1}} = \dfrac{n-1}{51} \). Вероятность вытащить туз2 из набора при дополнительном условии что это не туз1 равна \(\dfrac{1}{n-1}\). Тогда вероятность вытащить туз2 при условии что вытащили не туз1 равна \( \dfrac{n-1}{51} \cdot \dfrac{1}{n-1} = \dfrac{1}{51} \). Кроме туз2 есть ещё два туза в колоде, поэтому вероятность вытащить туз при условии что это не туз1 \(P(A|B_2)\) равна \(\dfrac{3}{51} \). Тогда \( P(A|B_2) \cdot P(B_2) = \dfrac{3}{51} \cdot \dfrac{n-1}{n} = \dfrac{3(n-1)}{51n} \) и окончательно 
\[  P(A) = P(A|B_1) \cdot P(B_1) + P(A|B_2) \cdot P(B_2) = \dfrac{1}{n} + \dfrac{3(n-1)}{51n} \]

В ходе решения продемонстрирован простой факт, вероятность того что конкретная карта из колоды из \(n\) карт попала в набор из \(m < n\) карт равна \( \dfrac{m}{n} \). Вероятность вытащить карту из набора равна \( \dfrac{1}{m} \). Получается что вероятность вытащить карту из набора равна вероятности вытащить карту из колоды. Это совпадает со случаем в задаче вытащить другую карту. Ищем вероятность вытащить другую карту при этом первая карта исключается из рассмотрения и сводится к вероятности вытащить туза из 51 карты с тремя тузами.


\paragraph{Решение Способ 2}
Известно что из колоды в 52 карты выбран набор из \(0<n<52\) карт и в этом наборе есть какой-то конкретный туз, обозначим его туз1.

Сколько всего существует таких наборов из \(n\) карт таких, что в них есть этот конкретный туз? К туз1 надо добавить \(n-1\) карт из 51. Таких наборов ровно \( \binom{51}{n-1} \).

Сколько существует таких наборов с ровно одним тузом (это будет туз1)? Таких наборов \( \binom{3}{0} \cdot \binom{48}{n-1} \). Вероятность выбрать туз и набора \(n\) карт если там один туз равна \( \dfrac{1}{n} \).

Сколько существует таких наборов с ровно двумя тузами (это будет туз1 и ещё один из трёх других тузов)? Таких наборов \( \binom{3}{1} \cdot \binom{48}{n-2} \). Вероятность выбрать туз и набора \(n\) карт если там два туза равна \( \dfrac{2}{n} \).

Сколько существует таких наборов с ровно тремя тузами (это будет туз1 и ещё два из трёх других тузов)? Таких наборов \( \binom{3}{2} \cdot \binom{48}{n-3} \). Вероятность выбрать туз и набора \(n\) карт если там три туза равна \( \dfrac{3}{n} \).

Сколько существует таких наборов с ровно четырьмя тузами (это будет туз1 и все три из трёх других тузов)? Таких наборов \( \binom{3}{3} \cdot \binom{48}{n-4} \). Вероятность выбрать туз и набора \(n\) карт если там четыре туза равна \( \dfrac{4}{n} \).

По формуле полной вероятности получаем что вероятность вытянуть туз будет равна сумме произведений вероятностей вытянуть туз из набора с фиксированным количеством i тузов на вероятность что в наборе ровно i тузов.

\[ P = \dfrac{1}{n \binom{51}{n-1}} \left( \binom{3}{0} \binom{48}{n-1} + 2 \binom{3}{1} \binom{48}{n-2} + 3 \binom{3}{2} \binom{48}{n-3} + 4 \binom{3}{3}  \binom{48}{n-4} \right) \]

Преобразуем дальше и получим

\[ P = \dfrac{1}{n \binom{51}{n-1}} \left( \binom{48}{n-1} + 6 \binom{48}{n-2} + 9 \binom{48}{n-3} + 4 \binom{48}{n-4} \right) \]

Преобразуя дальше можно получить

\( P = \dfrac{1}{n \cdot 51 \cdot 50 \cdot 49} \left( (52-n)(51-n)(50-n) + 6(52-n)(51-n)(n-1)\right)  + \dfrac{1}{n \cdot 51 \cdot 50 \cdot 49}\left(9 (52-n)(n-1)(n-2) + 4(n-1)(n-2)(n-3)  \right) \)
и далее повлучим
\( P = \dfrac{7350n+117600}{n \cdot 51 \cdot 50 \cdot 49 } = \dfrac{n+16}{17n} = \dfrac{1}{n} + \dfrac{3(n-1)}{51n} \) что совпадает с результатом решения способом 1.



\subsection{Независимые события}
\subsubsection*{Задача}
В мешке лежит 12 шаров для игры в петанк: 6 белых и 6 черных, причем 2 белых и 4 черных шара с насечкой.  Из мешка вынимают наугад один шар. 
Пусть A — событие, означающее, что вытащили белый шар, а B — событие, означающее, что вынули шар с насечкой. Найдите вероятности P(A), P(B) и \(P(A \cap B)\). Будут ли события A и B независимы? В ответе приведите разделенные пробелами найденные вероятности в виде обыкновенных дробей и одно из слов "зависимы" или "независимы" (без кавычек). Например, 1/2 2/3 3/4 независимы.
\paragraph{Решение}
Вероятность вытащить белый шар \(P(A) = \dfrac{1}{2}\). Вероятность вытащить шар с насечкой \(P(B) = \dfrac{1}{2}\). Пересечение событий A и B - вынули белый шар с насечкой, \(P(A \cap B) = \dfrac{1}{6}\). Вероятность пересечения событий A и B не равна произведению вероятностей событий A и B, значит события А и В зависимы.

\subsubsection*{Задача}
Из колоды, содержащей 52 карты, наугад извлекается одна карта. Событие A означает, что извлеченная карта является дамой, событие B — что извлечена карта пиковой масти. Найдите вероятности событий A, B и \(A \cap B\). Независимы ли события A и B? Решите эту же задачу при условии, что в колоде 54 карты, т.е. к стандартной 52-х карточной колоде добавлены два джокера. Джокер не имеет масти и не является дамой.
\paragraph{Решение}
Для колоды содержащей 52 карты вероятность вытащить даму \(P(A)= \dfrac{1}{13}\), вероятность вытащить карту пиковой масти \(P(B)= \dfrac{1}{4}\) и вероятность вытащить пиковую даму \(P(A \cap B)= \dfrac{1}{52}\). Вероятность пересечения событий равна произведению вероятностей событий, значит события независимы.

Для колоды содержащей 54 карты вероятность вытащить даму \(P(A)= \dfrac{4}{54}\), вероятность вытащить карту пиковой масти \(P(B)= \dfrac{13}{54}\) и вероятность вытащить пиковую даму \(P(A \cap B)= \dfrac{1}{54}\). Вероятность пересечения событий не равна произведению вероятностей событий, значит события зависимы.

\subsubsection*{Задача}
Подбрасываются две правильные игральные кости. Событие \(A_k\) при k=2,3,…,12 означает, что сумма очков на кубиках равна k, а событие \(B_n\) при n=1,2,…,6 означает, что на первом кубике выпало n очков. Для каких пар (k,n) события \(A_k\) и \(B_n\) независимы? В качестве ответа приведите все пары, разделенные пробелами. Например, (2,2) (4,3) (5,1).
\paragraph{Решение Способ 1} 
Найдём вероятности событий \(A_k\) и \(B_n\) при всех k и n.
\begin{itemize}
	\item \(A_2 = \dfrac{1}{36} \)
	\item \(A_3 = \dfrac{2}{36} \)
	\item \(A_4 = \dfrac{3}{36} \)
	\item \(A_5 = \dfrac{4}{36} \)
	\item \(A_6 = \dfrac{5}{36} \)
	\item \(A_7 = \dfrac{6}{36} \)
	\item \(A_8 = \dfrac{5}{36} \)
	\item \(A_9 = \dfrac{4}{36} \)
	\item \(A_10 = \dfrac{3}{36} \)
	\item \(A_11 = \dfrac{2}{36} \)
	\item \(A_12 = \dfrac{1}{36} \)
\end{itemize}

Для любого n=1,2,…,6 \(B_n = \dfrac{1}{6} \).

Для проверки независимости событий \(A_k \) и \(B_n\) надо для каждой пары (k,n)  проверить условие независимости \(P(A_k \cap B_n) = P(A_k)P(B_n)\).

\begin{itemize}
	\item \((2,1) -  P(A_2 \cap B_1) = \dfrac{1}{36} \neq P(A_2)P(B_1) = \dfrac{1}{36 \cdot 6} \) - зависимы
	\item \((3,1) -  P(A_3 \cap B_1) = \dfrac{1}{36} \neq P(A_2)P(B_1) = \dfrac{2}{36 \cdot 6} \) - зависимы
	\item \((4,1) -  P(A_4 \cap B_1) = \dfrac{1}{36} \neq P(A_2)P(B_1) = \dfrac{3}{36 \cdot 6} \) - зависимы
	\item \((5,1) -  P(A_5 \cap B_1) = \dfrac{1}{36} \neq P(A_2)P(B_1) = \dfrac{4}{36 \cdot 6} \) - зависимы
	\item \((6,1) -  P(A_6 \cap B_1) = \dfrac{1}{36} \neq P(A_2)P(B_1) = \dfrac{5}{36 \cdot 6} \) - зависимы
	\item \((7,1) -  P(A_7 \cap B_1) = \dfrac{1}{36} = P(A_2)P(B_1) = \dfrac{6}{36 \cdot 6} \) - независимы
	\item \((8,1) -  P(A_8 \cap B_1) = 0 \neq P(A_2)P(B_1) = \dfrac{5}{36 \cdot 6} \) - зависимы
	\item \((9,1) -  P(A_9 \cap B_1) = 0 \neq P(A_2)P(B_1) = \dfrac{4}{36 \cdot 6} \) - зависимы
	\item \((10,1) -  P(A_10 \cap B_1) = 0 \neq P(A_2)P(B_1) = \dfrac{3}{36 \cdot 6} \) - зависимы
	\item \((11,1) -  P(A_11 \cap B_1) = 0 \neq P(A_2)P(B_1) = \dfrac{2}{36 \cdot 6} \) - зависимы
	\item \((12,1) -  P(A_12 \cap B_1) = 0 \neq P(A_2)P(B_1) = \dfrac{1}{36 \cdot 6} \) - зависимы
\end{itemize}
\begin{itemize}
	\item \((2,2) -  P(A_2 \cap B_2) = 0 \neq P(A_2)P(B_2) = \dfrac{1}{36 \cdot 6} \) - зависимы
	\item \((3,2) -  P(A_3 \cap B_2) = \dfrac{1}{36} \neq P(A_2)P(B_2) = \dfrac{2}{36 \cdot 6} \) - зависимы
	\item \((4,2) -  P(A_4 \cap B_2) = \dfrac{1}{36} \neq P(A_2)P(B_2) = \dfrac{3}{36 \cdot 6} \) - зависимы
	\item \((5,2) -  P(A_5 \cap B_2) = \dfrac{1}{36} \neq P(A_2)P(B_2) = \dfrac{4}{36 \cdot 6} \) - зависимы
	\item \((6,2) -  P(A_6 \cap B_2) = \dfrac{1}{36} \neq P(A_2)P(B_2) = \dfrac{5}{36 \cdot 6} \) - зависимы
	\item \((7,2) -  P(A_7 \cap B_2) = \dfrac{1}{36} = P(A_2)P(B_2) = \dfrac{6}{36 \cdot 6} \) - независимы
	\item \((8,2) -  P(A_8 \cap B_2) = \dfrac{1}{36} \neq P(A_2)P(B_2) = \dfrac{5}{36 \cdot 6} \) - зависимы
	\item \((9,2) -  P(A_9 \cap B_2) = 0 \neq P(A_2)P(B_2) = \dfrac{4}{36 \cdot 6} \) - зависимы
	\item \((10,2) -  P(A_10 \cap B_2) = 0 \neq P(A_2)P(B_2) = \dfrac{3}{36 \cdot 6} \) - зависимы
	\item \((11,2) -  P(A_11 \cap B_2) = 0 \neq P(A_2)P(B_2) = \dfrac{2}{36 \cdot 6} \) - зависимы
	\item \((12,2) -  P(A_12 \cap B_2) = 0 \neq P(A_2)P(B_2) = \dfrac{1}{36 \cdot 6} \) - зависимы
\end{itemize}
\begin{itemize}
	\item \((2,3) -  P(A_2 \cap B_3) = 0 \neq P(A_2)P(B_3) = \dfrac{1}{36 \cdot 6} \) - зависимы
	\item \((3,3) -  P(A_3 \cap B_3) = 0 \neq P(A_2)P(B_3) = \dfrac{2}{36 \cdot 6} \) - зависимы
	\item \((4,3) -  P(A_4 \cap B_3) = \dfrac{1}{36} \neq P(A_2)P(B_3) = \dfrac{3}{36 \cdot 6} \) - зависимы
	\item \((5,3) -  P(A_5 \cap B_3) = \dfrac{1}{36} \neq P(A_2)P(B_3) = \dfrac{4}{36 \cdot 6} \) - зависимы
	\item \((6,3) -  P(A_6 \cap B_3) = \dfrac{1}{36} \neq P(A_2)P(B_3) = \dfrac{5}{36 \cdot 6} \) - зависимы
	\item \((7,3) -  P(A_7 \cap B_3) = \dfrac{1}{36} = P(A_2)P(B_3) = \dfrac{6}{36 \cdot 6} \) - независимы
	\item \((8,3) -  P(A_8 \cap B_3) = \dfrac{1}{36} \neq P(A_2)P(B_3) = \dfrac{5}{36 \cdot 6} \) - зависимы
	\item \((9,3) -  P(A_9 \cap B_3) = \dfrac{1}{36} \neq P(A_2)P(B_3) = \dfrac{4}{36 \cdot 6} \) - зависимы
	\item \((10,3) -  P(A_10 \cap B_3) = 0 \neq P(A_2)P(B_3) = \dfrac{3}{36 \cdot 6} \) - зависимы
	\item \((11,3) -  P(A_11 \cap B_3) = 0 \neq P(A_2)P(B_3) = \dfrac{2}{36 \cdot 6} \) - зависимы
	\item \((12,3) -  P(A_12 \cap B_3) = 0 \neq P(A_2)P(B_3) = \dfrac{1}{36 \cdot 6} \) - зависимы
\end{itemize}
\begin{itemize}
	\item \((2,4) -  P(A_2 \cap B_4) = 0 \neq P(A_2)P(B_4) = \dfrac{1}{36 \cdot 6} \) - зависимы
	\item \((3,4) -  P(A_3 \cap B_4) = 0 \neq P(A_2)P(B_4) = \dfrac{2}{36 \cdot 6} \) - зависимы
	\item \((4,4) -  P(A_4 \cap B_4) = 0 \neq P(A_2)P(B_4) = \dfrac{3}{36 \cdot 6} \) - зависимы
	\item \((5,4) -  P(A_5 \cap B_4) = \dfrac{1}{36} \neq P(A_2)P(B_4) = \dfrac{4}{36 \cdot 6} \) - зависимы
	\item \((6,4) -  P(A_6 \cap B_4) = \dfrac{1}{36} \neq P(A_2)P(B_4) = \dfrac{5}{36 \cdot 6} \) - зависимы
	\item \((7,4) -  P(A_7 \cap B_4) = \dfrac{1}{36} = P(A_2)P(B_4) = \dfrac{6}{36 \cdot 6} \) - независимы
	\item \((8,4) -  P(A_8 \cap B_4) = \dfrac{1}{36} \neq P(A_2)P(B_4) = \dfrac{5}{36 \cdot 6} \) - зависимы
	\item \((9,4) -  P(A_9 \cap B_4) = \dfrac{1}{36} \neq P(A_2)P(B_4) = \dfrac{4}{36 \cdot 6} \) - зависимы
	\item \((10,4) -  P(A_10 \cap B_4) = \dfrac{1}{36} \neq P(A_2)P(B_4) = \dfrac{3}{36 \cdot 6} \) - зависимы
	\item \((11,4) -  P(A_11 \cap B_4) = 0 \neq P(A_2)P(B_4) = \dfrac{2}{36 \cdot 6} \) - зависимы
	\item \((12,4) -  P(A_12 \cap B_4) = 0 \neq P(A_2)P(B_4) = \dfrac{1}{36 \cdot 6} \) - зависимы
\end{itemize}
\begin{itemize}
	\item \((2,5) -  P(A_2 \cap B_5) = 0 \neq P(A_2)P(B_5) = \dfrac{1}{36 \cdot 6} \) - зависимы
	\item \((3,5) -  P(A_3 \cap B_5) = 0 \neq P(A_2)P(B_5) = \dfrac{2}{36 \cdot 6} \) - зависимы
	\item \((4,5) -  P(A_4 \cap B_5) = 0 \neq P(A_2)P(B_5) = \dfrac{3}{36 \cdot 6} \) - зависимы
	\item \((5,5) -  P(A_5 \cap B_5) = 0 \neq P(A_2)P(B_5) = \dfrac{4}{36 \cdot 6} \) - зависимы
	\item \((6,5) -  P(A_6 \cap B_5) = \dfrac{1}{36} \neq P(A_2)P(B_5) = \dfrac{5}{36 \cdot 6} \) - зависимы
	\item \((7,5) -  P(A_7 \cap B_5) = \dfrac{1}{36} = P(A_2)P(B_5) = \dfrac{6}{36 \cdot 6} \) - независимы
	\item \((8,5) -  P(A_8 \cap B_5) = \dfrac{1}{36} \neq P(A_2)P(B_5) = \dfrac{5}{36 \cdot 6} \) - зависимы
	\item \((9,5) -  P(A_9 \cap B_5) = \dfrac{1}{36} \neq P(A_2)P(B_5) = \dfrac{4}{36 \cdot 6} \) - зависимы
	\item \((10,5) -  P(A_10 \cap B_5) = \dfrac{1}{36} \neq P(A_2)P(B_5) = \dfrac{3}{36 \cdot 6} \) - зависимы
	\item \((11,5) -  P(A_11 \cap B_5) = \dfrac{1}{36} \neq P(A_2)P(B_5) = \dfrac{2}{36 \cdot 6} \) - зависимы
	\item \((12,5) -  P(A_12 \cap B_5) = 0 \neq P(A_2)P(B_5) = \dfrac{1}{36 \cdot 6} \) - зависимы
\end{itemize}
\begin{itemize}
	\item \((2,6) -  P(A_2 \cap B_6) = 0 \neq P(A_2)P(B_6) = \dfrac{1}{36 \cdot 6} \) - зависимы
	\item \((3,6) -  P(A_3 \cap B_6) = 0 \neq P(A_2)P(B_6) = \dfrac{2}{36 \cdot 6} \) - зависимы
	\item \((4,6) -  P(A_4 \cap B_6) = 0 \neq P(A_2)P(B_6) = \dfrac{3}{36 \cdot 6} \) - зависимы
	\item \((5,6) -  P(A_5 \cap B_6) = 0 \neq P(A_2)P(B_6) = \dfrac{4}{36 \cdot 6} \) - зависимы
	\item \((6,6) -  P(A_6 \cap B_6) = 0 \neq P(A_2)P(B_6) = \dfrac{5}{36 \cdot 6} \) - зависимы
	\item \((7,6) -  P(A_7 \cap B_6) = \dfrac{1}{36} = P(A_2)P(B_6) = \dfrac{6}{36 \cdot 6} \) - независимы
	\item \((8,6) -  P(A_8 \cap B_6) = \dfrac{1}{36} \neq P(A_2)P(B_6) = \dfrac{5}{36 \cdot 6} \) - зависимы
	\item \((9,6) -  P(A_9 \cap B_6) = \dfrac{1}{36} \neq P(A_2)P(B_6) = \dfrac{4}{36 \cdot 6} \) - зависимы
	\item \((10,6) -  P(A_10 \cap B_6) = \dfrac{1}{36} \neq P(A_2)P(B_6) = \dfrac{3}{36 \cdot 6} \) - зависимы
	\item \((11,6) -  P(A_11 \cap B_6) = \dfrac{1}{36} \neq P(A_2)P(B_6) = \dfrac{2}{36 \cdot 6} \) - зависимы
	\item \((12,6) -  P(A_12 \cap B_6) = \dfrac{1}{36} \neq P(A_2)P(B_6) = \dfrac{1}{36 \cdot 6} \) - зависимы
\end{itemize}

Итого независимы события при (k,n): (7,1),(7,2),(7,3),(7,4),(7,5),(7,6).

\paragraph{Решение Способ 2} 
Напишем вероятности событий \(A_k\) и \(B_n\): \(P(B_n)= \dfrac{1}{6}\), \(P(A_k)=\dfrac{k-1}{36}\) при k=2,3,…,7 и \(P(A_k)=\dfrac{13-k}{36}\) при k=8,9,…,12. Вероятность события \(A_k \cap B_n\) также легко находится, поскольку это событие означает, что на первом кубике выпало n, а в сумме выпало k. Если \(k \leqslant n\), то это невозможное событие и его вероятность равна нулю. Тогда события \(A_k\) и \(B_n\) не могут быть независимыми, ибо \(P(A_k)P(B_n) > 0=P(A_k \cap B_n)\). Если же k>n, то это означает, что на втором кубике выпало k-n и вероятность выпадения такой пары очков равна \(\dfrac{1}{36}\). Установим когда k>n и \(P(A_k)P(B_n)=P(A_k \cap B_n)\). Рассмотрим случай \(k \leqslant 7\), тогда \(\dfrac{1}{36}=\dfrac{k-1}{36} \cdot \dfrac{1}{6}\) и, значит, k=7 и n — любое. Далее рассмотрим случай \(k \geqslant 8\), тогда \(\dfrac{1}{36}=\dfrac{13-k}{36} \cdot \dfrac{1}{6}\) и, значит, 13-k=6, что для \(k \geqslant 8\) невозможно. Отсюда находим ответ: k=7, n — любое число от 1 до 6.

\subsubsection*{Задача}
Бросаются две игральные кости. Рассмотрим три события: A — на первой кости выпало нечётное число очков, B — на второй кости выпало нечётное число очков, C — сумма очков на обеих костях нечётна. 

Бросаются три игральные кости. Событие X
состоит в том, что одинаковое число очков выпало на первой и второй костях, Y — одинаковое число очков на второй и третьей костях, Z — на первой и третьей.

Отметьте верные утверждения.
\begin{itemize}
    \item события \(A\) и \(\overline{A}\) независимы: неверно
    \item события \(A\) и \(\overline{B}\) независимы: верно 
    \item события \(A\),  \(B\) и \(C\) попарно независимы: верно
    \item события \(A\),  \(B\) и \(C\) независимы в совокупности: неверно 
    \item события \(X\),  \(Y\) и \(Z\) попарно независимы: верно 
    \item события \(X\),  \(Y\) и \(Z\) независимы в совокупности: неверно  
\end{itemize}

\paragraph{Решение}


\subsection{Схема Бернулли}
\subsubsection*{Задача}
Даны иррациональное число \(p \in (0,1)\) и натуральное число \(n\). Сравнив вероятности \(P(A_k)\) при соседних \(k\) (события \(A_k\) определены в предыдущем видео), найдите при каком \(k\) величина \(P(A_k)\) будет наибольшей. В качестве ответа приведите указанный номер \(k\). Для записи формулы пригодится функция \(floor(x)\), обозначающая целую часть числа \(x\).
\paragraph{Решение}
Событие \(A_k\) - в схеме Бернулли ровно \(k\) успехов. Вероятность события \(A_k\) равна \(P(A_k)=\binom{n}{k}p^k(q)^{n-k}\) где \(q=1-p\).



\subsection{Краткие сведения из математического анализа}


\section{Элементарная теория вероятностей: случайные величины}




\end{document} % Конец текста.

